% Options for packages loaded elsewhere
\PassOptionsToPackage{unicode}{hyperref}
\PassOptionsToPackage{hyphens}{url}
%
\documentclass[
]{article}
\usepackage{lmodern}
\usepackage{amssymb,amsmath}
\usepackage{ifxetex,ifluatex}
\ifnum 0\ifxetex 1\fi\ifluatex 1\fi=0 % if pdftex
  \usepackage[T1]{fontenc}
  \usepackage[utf8]{inputenc}
  \usepackage{textcomp} % provide euro and other symbols
\else % if luatex or xetex
  \usepackage{unicode-math}
  \defaultfontfeatures{Scale=MatchLowercase}
  \defaultfontfeatures[\rmfamily]{Ligatures=TeX,Scale=1}
\fi
% Use upquote if available, for straight quotes in verbatim environments
\IfFileExists{upquote.sty}{\usepackage{upquote}}{}
\IfFileExists{microtype.sty}{% use microtype if available
  \usepackage[]{microtype}
  \UseMicrotypeSet[protrusion]{basicmath} % disable protrusion for tt fonts
}{}
\makeatletter
\@ifundefined{KOMAClassName}{% if non-KOMA class
  \IfFileExists{parskip.sty}{%
    \usepackage{parskip}
  }{% else
    \setlength{\parindent}{0pt}
    \setlength{\parskip}{6pt plus 2pt minus 1pt}}
}{% if KOMA class
  \KOMAoptions{parskip=half}}
\makeatother
\usepackage{xcolor}
\IfFileExists{xurl.sty}{\usepackage{xurl}}{} % add URL line breaks if available
\IfFileExists{bookmark.sty}{\usepackage{bookmark}}{\usepackage{hyperref}}
\hypersetup{
  pdftitle={Agrupamiento barrios por accidentalidad},
  pdfauthor={Carlos Bolivar},
  hidelinks,
  pdfcreator={LaTeX via pandoc}}
\urlstyle{same} % disable monospaced font for URLs
\usepackage[margin=1in]{geometry}
\usepackage{color}
\usepackage{fancyvrb}
\newcommand{\VerbBar}{|}
\newcommand{\VERB}{\Verb[commandchars=\\\{\}]}
\DefineVerbatimEnvironment{Highlighting}{Verbatim}{commandchars=\\\{\}}
% Add ',fontsize=\small' for more characters per line
\usepackage{framed}
\definecolor{shadecolor}{RGB}{248,248,248}
\newenvironment{Shaded}{\begin{snugshade}}{\end{snugshade}}
\newcommand{\AlertTok}[1]{\textcolor[rgb]{0.94,0.16,0.16}{#1}}
\newcommand{\AnnotationTok}[1]{\textcolor[rgb]{0.56,0.35,0.01}{\textbf{\textit{#1}}}}
\newcommand{\AttributeTok}[1]{\textcolor[rgb]{0.77,0.63,0.00}{#1}}
\newcommand{\BaseNTok}[1]{\textcolor[rgb]{0.00,0.00,0.81}{#1}}
\newcommand{\BuiltInTok}[1]{#1}
\newcommand{\CharTok}[1]{\textcolor[rgb]{0.31,0.60,0.02}{#1}}
\newcommand{\CommentTok}[1]{\textcolor[rgb]{0.56,0.35,0.01}{\textit{#1}}}
\newcommand{\CommentVarTok}[1]{\textcolor[rgb]{0.56,0.35,0.01}{\textbf{\textit{#1}}}}
\newcommand{\ConstantTok}[1]{\textcolor[rgb]{0.00,0.00,0.00}{#1}}
\newcommand{\ControlFlowTok}[1]{\textcolor[rgb]{0.13,0.29,0.53}{\textbf{#1}}}
\newcommand{\DataTypeTok}[1]{\textcolor[rgb]{0.13,0.29,0.53}{#1}}
\newcommand{\DecValTok}[1]{\textcolor[rgb]{0.00,0.00,0.81}{#1}}
\newcommand{\DocumentationTok}[1]{\textcolor[rgb]{0.56,0.35,0.01}{\textbf{\textit{#1}}}}
\newcommand{\ErrorTok}[1]{\textcolor[rgb]{0.64,0.00,0.00}{\textbf{#1}}}
\newcommand{\ExtensionTok}[1]{#1}
\newcommand{\FloatTok}[1]{\textcolor[rgb]{0.00,0.00,0.81}{#1}}
\newcommand{\FunctionTok}[1]{\textcolor[rgb]{0.00,0.00,0.00}{#1}}
\newcommand{\ImportTok}[1]{#1}
\newcommand{\InformationTok}[1]{\textcolor[rgb]{0.56,0.35,0.01}{\textbf{\textit{#1}}}}
\newcommand{\KeywordTok}[1]{\textcolor[rgb]{0.13,0.29,0.53}{\textbf{#1}}}
\newcommand{\NormalTok}[1]{#1}
\newcommand{\OperatorTok}[1]{\textcolor[rgb]{0.81,0.36,0.00}{\textbf{#1}}}
\newcommand{\OtherTok}[1]{\textcolor[rgb]{0.56,0.35,0.01}{#1}}
\newcommand{\PreprocessorTok}[1]{\textcolor[rgb]{0.56,0.35,0.01}{\textit{#1}}}
\newcommand{\RegionMarkerTok}[1]{#1}
\newcommand{\SpecialCharTok}[1]{\textcolor[rgb]{0.00,0.00,0.00}{#1}}
\newcommand{\SpecialStringTok}[1]{\textcolor[rgb]{0.31,0.60,0.02}{#1}}
\newcommand{\StringTok}[1]{\textcolor[rgb]{0.31,0.60,0.02}{#1}}
\newcommand{\VariableTok}[1]{\textcolor[rgb]{0.00,0.00,0.00}{#1}}
\newcommand{\VerbatimStringTok}[1]{\textcolor[rgb]{0.31,0.60,0.02}{#1}}
\newcommand{\WarningTok}[1]{\textcolor[rgb]{0.56,0.35,0.01}{\textbf{\textit{#1}}}}
\usepackage{graphicx,grffile}
\makeatletter
\def\maxwidth{\ifdim\Gin@nat@width>\linewidth\linewidth\else\Gin@nat@width\fi}
\def\maxheight{\ifdim\Gin@nat@height>\textheight\textheight\else\Gin@nat@height\fi}
\makeatother
% Scale images if necessary, so that they will not overflow the page
% margins by default, and it is still possible to overwrite the defaults
% using explicit options in \includegraphics[width, height, ...]{}
\setkeys{Gin}{width=\maxwidth,height=\maxheight,keepaspectratio}
% Set default figure placement to htbp
\makeatletter
\def\fps@figure{htbp}
\makeatother
\setlength{\emergencystretch}{3em} % prevent overfull lines
\providecommand{\tightlist}{%
  \setlength{\itemsep}{0pt}\setlength{\parskip}{0pt}}
\setcounter{secnumdepth}{-\maxdimen} % remove section numbering

\title{Agrupamiento barrios por accidentalidad}
\author{Carlos Bolivar}
\date{14/11/2020}

\begin{document}
\maketitle

Lectura de datos

\begin{Shaded}
\begin{Highlighting}[]
\NormalTok{accidentes <-}\StringTok{ }\KeywordTok{read.csv}\NormalTok{(}\StringTok{"../dataset/accidentes_medellin.csv"}\NormalTok{,}\DataTypeTok{header =} \OtherTok{TRUE}\NormalTok{,}\DataTypeTok{encoding=}\StringTok{'UTF-8'}\NormalTok{)}
\end{Highlighting}
\end{Shaded}

Librerías

\begin{Shaded}
\begin{Highlighting}[]
\KeywordTok{library}\NormalTok{(}\StringTok{"dplyr"}\NormalTok{) }\CommentTok{## load}
\end{Highlighting}
\end{Shaded}

\begin{verbatim}
## 
## Attaching package: 'dplyr'
\end{verbatim}

\begin{verbatim}
## The following objects are masked from 'package:stats':
## 
##     filter, lag
\end{verbatim}

\begin{verbatim}
## The following objects are masked from 'package:base':
## 
##     intersect, setdiff, setequal, union
\end{verbatim}

\begin{Shaded}
\begin{Highlighting}[]
\KeywordTok{library}\NormalTok{(}\StringTok{"factoextra"}\NormalTok{)}
\end{Highlighting}
\end{Shaded}

\begin{verbatim}
## Loading required package: ggplot2
\end{verbatim}

\begin{verbatim}
## Welcome! Want to learn more? See two factoextra-related books at https://goo.gl/ve3WBa
\end{verbatim}

\begin{Shaded}
\begin{Highlighting}[]
\KeywordTok{library}\NormalTok{(}\StringTok{"clValid"}\NormalTok{)}
\end{Highlighting}
\end{Shaded}

\begin{verbatim}
## Loading required package: cluster
\end{verbatim}

\begin{Shaded}
\begin{Highlighting}[]
\KeywordTok{library}\NormalTok{(}\StringTok{"sf"}\NormalTok{)}
\end{Highlighting}
\end{Shaded}

\begin{verbatim}
## Linking to GEOS 3.8.0, GDAL 3.0.4, PROJ 6.3.1
\end{verbatim}

\begin{Shaded}
\begin{Highlighting}[]
\KeywordTok{library}\NormalTok{(}\StringTok{"ggplot2"}\NormalTok{)}
\KeywordTok{set.seed}\NormalTok{(}\DecValTok{31}\NormalTok{)}
\end{Highlighting}
\end{Shaded}

\hypertarget{creacion-de-variables-convenientes-para-la-agrupacion}{%
\subsection{Creacion de variables convenientes para la
agrupacion}\label{creacion-de-variables-convenientes-para-la-agrupacion}}

CF -\textgreater{} Cantidad de incidentes días festivos. ES
-\textgreater{} Cantidad de incidentes entre semana. FS -\textgreater{}
Cantidad de incidentes fin de semana. CA -\textgreater{} Cantidad de
incidentes de clase `atropello'. CC -\textgreater{} Cantidad de
incidentes de clase `choque'. CH -\textgreater{} Cantidad de incidentes
de gravedad `herido'. CM -\textgreater{} Cantidad de incidentes de
gravedad `muerto'.

\begin{Shaded}
\begin{Highlighting}[]
\NormalTok{  accidentes_vars <-}\StringTok{ }\NormalTok{accidentes }\OperatorTok
\StringTok{                }\KeywordTok{select}\NormalTok{(barrio, festivo, dia_nombre, clase, gravedad) }\OperatorTok
\StringTok{                }\KeywordTok{group_by}\NormalTok{(barrio) }\OperatorTok
\StringTok{                }\KeywordTok{summarise}\NormalTok{(}\DataTypeTok{CF =} \KeywordTok{sum}\NormalTok{(festivo }\OperatorTok{==}\StringTok{ "True"}\NormalTok{),}
                       \DataTypeTok{ES =} \KeywordTok{sum}\NormalTok{(dia_nombre }\OperatorTok\StringTok{ }\KeywordTok{c}\NormalTok{(}\StringTok{'lunes'}\NormalTok{, }\StringTok{'martes'}\NormalTok{, }\StringTok{'miércoles'}\NormalTok{, }\StringTok{'jueves'}\NormalTok{,}\StringTok{'viernes'}\NormalTok{) }\OperatorTok{&}\StringTok{ }\NormalTok{festivo}\OperatorTok{==}\StringTok{"False"}\NormalTok{),}
                       \DataTypeTok{FS =} \KeywordTok{sum}\NormalTok{(dia_nombre }\OperatorTok\StringTok{ }\KeywordTok{c}\NormalTok{(}\StringTok{'sábado'}\NormalTok{,}\StringTok{'domingo'}\NormalTok{) }\OperatorTok{&}\StringTok{ }\NormalTok{festivo}\OperatorTok{==}\StringTok{"False"}\NormalTok{),}
                       \DataTypeTok{CA =} \KeywordTok{sum}\NormalTok{(clase }\OperatorTok{==}\StringTok{ 'atropello'}\NormalTok{),}
                       \DataTypeTok{CC =} \KeywordTok{sum}\NormalTok{(clase }\OperatorTok{==}\StringTok{ 'choque'}\NormalTok{),}
                       \DataTypeTok{CH =} \KeywordTok{sum}\NormalTok{(gravedad }\OperatorTok{==}\StringTok{ 'herido'}\NormalTok{),}
                       \DataTypeTok{CM =} \KeywordTok{sum}\NormalTok{(gravedad }\OperatorTok{==}\StringTok{ 'muerto'}\NormalTok{),}
                       \DataTypeTok{.groups =} \StringTok{'drop'}
\NormalTok{                       )}
\NormalTok{  accidentes_vars <-}\StringTok{ }\KeywordTok{data.frame}\NormalTok{(accidentes_vars)}
  \KeywordTok{row.names}\NormalTok{(accidentes_vars) <-}\StringTok{ }\NormalTok{accidentes_vars}\OperatorTok{$}\NormalTok{barrio}
\NormalTok{  accidentes_vars[}\DecValTok{1}\NormalTok{] <-}\StringTok{ }\OtherTok{NULL}
  \KeywordTok{head}\NormalTok{(accidentes_vars)}
\end{Highlighting}
\end{Shaded}

\begin{verbatim}
##                      CF  ES  FS  CA  CC  CH CM
## aldea pablo vi        5  45  33  27  35  64  2
## alejandría           13 411  75  14 434 132  1
## alejandro echavarría 47 556 237  98 482 595  2
## alfonso lópez        51 777 319 132 607 772  6
## altamira             17 597 162  50 466 504  2
## altavista            11 241 101  62 206 214  2
\end{verbatim}

\hypertarget{preprocesamiento-y-validaciuxf3n}{%
\subsection{Preprocesamiento y
validación}\label{preprocesamiento-y-validaciuxf3n}}

Escalar los datos

\begin{Shaded}
\begin{Highlighting}[]
\NormalTok{accidentes_sc <-}\StringTok{ }\KeywordTok{scale}\NormalTok{(accidentes_vars)}
\end{Highlighting}
\end{Shaded}

Se calcula el estadístico de Hopkins para verificar si los datos tienen
tendencia a agruparse. Si el estadistico es mayor a 0.5, esto quiere
decir que el conjunto de datos es significativamente agrupable

\begin{Shaded}
\begin{Highlighting}[]
\NormalTok{res <-}\StringTok{ }\KeywordTok{get_clust_tendency}\NormalTok{(accidentes_sc, }\DataTypeTok{n =} \KeywordTok{nrow}\NormalTok{(accidentes_sc)}\OperatorTok{-}\DecValTok{1}\NormalTok{, }\DataTypeTok{graph =} \OtherTok{FALSE}\NormalTok{)}
\NormalTok{res}\OperatorTok{$}\NormalTok{hopkins_stat}
\end{Highlighting}
\end{Shaded}

\begin{verbatim}
## [1] 0.8876382
\end{verbatim}

Se calcula el K óptimo para kmeans

\begin{Shaded}
\begin{Highlighting}[]
\KeywordTok{fviz_nbclust}\NormalTok{(accidentes_sc, kmeans, }\DataTypeTok{method =} \StringTok{"wss"}\NormalTok{, }\DataTypeTok{k.max =} \DecValTok{24}\NormalTok{) }\OperatorTok{+}\StringTok{ }\KeywordTok{theme_minimal}\NormalTok{() }\OperatorTok{+}\StringTok{ }\KeywordTok{ggtitle}\NormalTok{(}\StringTok{"Elbow method"}\NormalTok{)}
\end{Highlighting}
\end{Shaded}

\includegraphics{cluster_files/figure-latex/unnamed-chunk-6-1.pdf}

Se considera \(k=3\) o \(k=4\) como numero de clusters apropiado.

\hypertarget{eleccion-de-algoritmo-de-agrupacion.}{%
\subsection{Eleccion de algoritmo de
agrupacion.}\label{eleccion-de-algoritmo-de-agrupacion.}}

Se utiliza un estadístico de validación, para estimar qué algoritmo
puede realizar la mejor estimación de los clusters. Los algoritmos
considerados son hierarchical, kmeans, pam y clara.

\begin{Shaded}
\begin{Highlighting}[]
\NormalTok{intern <-}\StringTok{ }\KeywordTok{clValid}\NormalTok{(accidentes_sc, }\DataTypeTok{nClust =} \DecValTok{3}\OperatorTok{:}\DecValTok{4}\NormalTok{, }
              \DataTypeTok{clMethods =} \KeywordTok{c}\NormalTok{(}\StringTok{"hierarchical"}\NormalTok{,}\StringTok{"kmeans"}\NormalTok{,}\StringTok{"pam"}\NormalTok{,}\StringTok{'clara'}\NormalTok{),}
              \DataTypeTok{validation =} \StringTok{"internal"}\NormalTok{)}
\CommentTok{# Summary}
\KeywordTok{summary}\NormalTok{(intern)}
\end{Highlighting}
\end{Shaded}

\begin{verbatim}
## 
## Clustering Methods:
##  hierarchical kmeans pam clara 
## 
## Cluster sizes:
##  3 4 
## 
## Validation Measures:
##                                  3       4
##                                           
## hierarchical Connectivity  11.5675 18.8095
##              Dunn           0.1380  0.2074
##              Silhouette     0.6839  0.5983
## kmeans       Connectivity  11.5675 44.8353
##              Dunn           0.1380  0.0266
##              Silhouette     0.6839  0.4739
## pam          Connectivity  31.8651 75.9147
##              Dunn           0.0155  0.0086
##              Silhouette     0.4315  0.3427
## clara        Connectivity  45.9532 81.8036
##              Dunn           0.0170  0.0096
##              Silhouette     0.4444  0.3448
## 
## Optimal Scores:
## 
##              Score   Method       Clusters
## Connectivity 11.5675 hierarchical 3       
## Dunn          0.2074 hierarchical 4       
## Silhouette    0.6839 hierarchical 3
\end{verbatim}

Con los resultados obtenidos, se puede concluír que kmeans o
hierarchical con \(k=3\) son los algoritmos de agrupamiento óptimo. Se
elije Kmeans.

\begin{Shaded}
\begin{Highlighting}[]
\NormalTok{km.res <-}\StringTok{ }\KeywordTok{kmeans}\NormalTok{(accidentes_sc, }\DecValTok{3}\NormalTok{, }\DataTypeTok{nstart =} \DecValTok{25}\NormalTok{)}
\end{Highlighting}
\end{Shaded}

Cantidad de barrios por grupo

\begin{Shaded}
\begin{Highlighting}[]
\KeywordTok{table}\NormalTok{(km.res[}\DecValTok{1}\NormalTok{])}
\end{Highlighting}
\end{Shaded}

\begin{verbatim}
## 
##   1   2   3 
## 159  22  88
\end{verbatim}

Se agrega la columna de grupo a los datos de accidentes

\begin{Shaded}
\begin{Highlighting}[]
\NormalTok{accidentes_vars}\OperatorTok{$}\NormalTok{grupo <-}\StringTok{ }\NormalTok{km.res}\OperatorTok{$}\NormalTok{cluster}
\end{Highlighting}
\end{Shaded}

Nombres para cada cluster

\begin{Shaded}
\begin{Highlighting}[]
\NormalTok{grupos_hclust_lb <-}\StringTok{ }\KeywordTok{ifelse}\NormalTok{(km.res}\OperatorTok{$}\NormalTok{cluster}\OperatorTok{==}\DecValTok{1}\NormalTok{,}\StringTok{"Riesgo menor"}\NormalTok{,}
                           \KeywordTok{ifelse}\NormalTok{(km.res}\OperatorTok{$}\NormalTok{cluster}\OperatorTok{==}\DecValTok{2}\NormalTok{,}\StringTok{"Riesgo mayor"}\NormalTok{, }\StringTok{"Riesgo medio"}\NormalTok{))}
\end{Highlighting}
\end{Shaded}

Medias:

\begin{Shaded}
\begin{Highlighting}[]
\KeywordTok{aggregate}\NormalTok{(}\KeywordTok{cbind}\NormalTok{(CF,ES,FS,CA,CC,CH,CM)}\OperatorTok{~}\NormalTok{grupos_hclust_lb,}\DataTypeTok{data=}\NormalTok{accidentes_vars,}\DataTypeTok{FUN=}\NormalTok{mean)}
\end{Highlighting}
\end{Shaded}

\begin{verbatim}
##   grupos_hclust_lb       CF        ES        FS        CA        CC        CH
## 1     Riesgo mayor 60.81818 2398.5909 634.90909 251.22727 2286.0000 1447.5000
## 2     Riesgo medio 32.60227  734.8182 249.73864 100.89773  667.0568  588.1591
## 3     Riesgo menor 10.33962  203.5597  75.30189  34.67296  183.9434  168.1824
##          CM
## 1 19.590909
## 2  5.625000
## 3  1.603774
\end{verbatim}

\hypertarget{lectura-de-mapa-y-relacion-con-los-resultados-obtenidos}{%
\subsection{Lectura de mapa y relacion con los resultados
obtenidos}\label{lectura-de-mapa-y-relacion-con-los-resultados-obtenidos}}

Lectura de mapa.

\begin{Shaded}
\begin{Highlighting}[]
\KeywordTok{options}\NormalTok{(}\DataTypeTok{scipen =} \DecValTok{999}\NormalTok{)}
\NormalTok{medellin_shape <-}\StringTok{ }\KeywordTok{st_read}\NormalTok{(}\StringTok{"../shapes/Barrio_Vereda/BarrioVereda_2014.shp"}\NormalTok{, }\DataTypeTok{stringsAsFactors=}\OtherTok{FALSE}\NormalTok{)}
\end{Highlighting}
\end{Shaded}

\begin{verbatim}
## Reading layer `BarrioVereda_2014' from data source `C:\Users\cdbol\Documents\UniversidadRepositorios\AccidentalidadMedellin\shapes\Barrio_Vereda\BarrioVereda_2014.shp' using driver `ESRI Shapefile'
## Simple feature collection with 332 features and 12 fields
## geometry type:  MULTIPOLYGON
## dimension:      XY
## bbox:           xmin: 818288.4 ymin: 1173484 xmax: 845676.7 ymax: 1196930
\end{verbatim}

\begin{verbatim}
## Warning in CPL_crs_parameters(x): GDAL Error 1: PROJ: proj_as_proj_string:
## Unsupported conversion method: IGAC_Plano_Cartesiano

## Warning in CPL_crs_parameters(x): GDAL Error 1: PROJ: proj_as_proj_string:
## Unsupported conversion method: IGAC_Plano_Cartesiano
\end{verbatim}

\begin{verbatim}
## projected CRS:  MAGNA-SIRGAS / Medellin urban grid
\end{verbatim}

\begin{Shaded}
\begin{Highlighting}[]
\NormalTok{medellin_shape}\OperatorTok{$}\NormalTok{NOMBRE <-}\StringTok{ }\KeywordTok{tolower}\NormalTok{(medellin_shape}\OperatorTok{$}\NormalTok{NOMBRE)}
\end{Highlighting}
\end{Shaded}

Relación de los datos agrupados con el mapa

\begin{Shaded}
\begin{Highlighting}[]
\NormalTok{accidentes_vars}\OperatorTok{$}\NormalTok{NOMBRE <-}\StringTok{ }\KeywordTok{row.names}\NormalTok{(accidentes_vars)}
\NormalTok{map_and_data <-}\StringTok{ }\KeywordTok{inner_join}\NormalTok{(medellin_shape, accidentes_vars)}
\end{Highlighting}
\end{Shaded}

\begin{verbatim}
## Joining, by = "NOMBRE"
\end{verbatim}

\hypertarget{mapa-de-barrios-y-veredas-medelluxedn-agrupados-por-accidentalidad}{%
\subsection{Mapa de barrios y veredas Medellín agrupados por
accidentalidad}\label{mapa-de-barrios-y-veredas-medelluxedn-agrupados-por-accidentalidad}}

\begin{Shaded}
\begin{Highlighting}[]
\KeywordTok{ggplot}\NormalTok{(map_and_data)}\OperatorTok{+}
\StringTok{  }\KeywordTok{geom_sf}\NormalTok{(}\KeywordTok{aes}\NormalTok{(}\DataTypeTok{fill =} \KeywordTok{factor}\NormalTok{(grupo) )) }\OperatorTok{+}
\StringTok{  }\KeywordTok{scale_color_discrete}\NormalTok{(}\StringTok{"Cluster"}\NormalTok{)}\OperatorTok{+}
\StringTok{  }\KeywordTok{theme_bw}\NormalTok{()}\OperatorTok{+}
\StringTok{  }\KeywordTok{scale_fill_manual}\NormalTok{(}\DataTypeTok{values =} \KeywordTok{c}\NormalTok{(}\StringTok{"#daedd2"}\NormalTok{, }\StringTok{"#469536"}\NormalTok{,}\StringTok{"#a8cc99"}\NormalTok{), }\DataTypeTok{name=} \StringTok{"Accidentes por barrios Medellín"}\NormalTok{,}
                    \DataTypeTok{labels =} \KeywordTok{c}\NormalTok{(}\StringTok{"Riesgo bajo"}\NormalTok{, }\StringTok{"Riesgo alto"}\NormalTok{, }\StringTok{"Riesgo medio"}\NormalTok{))}\OperatorTok{+}\StringTok{ }
\StringTok{  }\KeywordTok{theme}\NormalTok{(}\DataTypeTok{legend.position =} \StringTok{"right"}\NormalTok{)}
\end{Highlighting}
\end{Shaded}

\includegraphics{cluster_files/figure-latex/unnamed-chunk-15-1.pdf} \#\#
Características espaciales por grupo

\emph{Genaral:} es notable que el riesgo de accidente de tráficio por
barrio aumenta a medida que se va acercando al centro del municipio,
excepto en el caso del corriegimiento de San Cristobal. Cabe reclacar
que lo que se menciona como riesgo, esta enfocado a la probabilidad de
sufrir un incidente (choque o atropello) en un barrio y resultar herido
o muerto.

\emph{Riesgo bajo:} este conjunto está conformado en su mayoría por
algunas veredas en el occidente del municipio, barrios pertenecientes a
las comunas San Javier, La América, El Poblado, Buenos Aires, Villa
Hermosa, Santa Cruz y El Popular. Esto puede ser debido a que estos
lugares son espacios recidenciales, por lo cual los trayectos entre
calles es reducido y las velocidades alcanzadas por los medios de
transporte son moderadas, además de que los niveles de tráfico son mas
reducidos. Todos estos factores hacen que este conjunto se considere
como un riesgo bajo y se ve evidenciado con los resultados obtenidos.

\emph{Riesgo medio:} este conjunto está conformado por los barrios que
estan entre los extremos del municipio y los del centro. Algunos barrios
que hacen parte de las comunas 12 de Octubre, Castilla, Aranjuez,
Laureles Estadio, La Candelaria, Manrique, Belén, Guayabal y El Poblado.
Las razones por las que se presenta esto, es por que estos barrios
conectan las zonas recidenciales con las comerciales y ayudan a llegan a
las vias principles del municipio; debido a esto, hay un aumento en el
tráfico, velocidades y por ende en la probabilidad de tener un
accidente.

\emph{Riesgo alto:} como anteriormente se mencionó este conjunto está
ubicado en el centro del municipio, ya que ahí está ubicada la zona
comercial y las principales vías para entrar y salir del municipio, lo
que conlleva a al máximo tráfico. además que se tiene que considerar que
en estas vías entran vehiculos que ocupan mas espacio y son mas pesados,
tales como, camiones, mulas, buses, etc. si a esto le agregamos altas
velocidades, no solo la probabilidad de accidente se hace más grande,
así mismo aumenta la probabilidad de que una o varias personas resulten
heridas o en el peor de los casos muerta en algún accidente.

\end{document}
